\documentclass[conference]{IEEEtran}
% \documentclass{report}

\usepackage{cite}
\usepackage{graphicx}
\usepackage{algorithmic}
\usepackage{amsmath}
\usepackage{amsfonts} % for "\mathbb" macro
\usepackage[caption=false,font=normalsize,labelfont=sf,textfont=sf]{subfig}
\usepackage{url}
\usepackage{hyperref}
\usepackage{float}
\usepackage{listings}
% Customizing lstlisting to mimic verbatim
\lstset{
  basicstyle=\ttfamily,
  breaklines=true, % Allows line breaks
  postbreak=\mbox{\space}, % Optional: mark line breaks
  escapeinside={(@}{@)}, % Define escape characters
}

\newcommand{\N}{\mathbb{N}}
\newcommand{\Z}{\mathbb{Z}}
\newcommand{\Q}{\mathbb{Q}}
\newcommand{\R}{\mathbb{R}}
\newcommand{\C}{\mathbb{C}}


\begin{document}
\title{Phisical Layer Security using Reconfigurable Intelligent Surfaces in Multiple Users and Multiple Reflections MIMO Communications}


% author names and affiliations
% use a multiple column layout for up to three different
% affiliations
\author{
  \IEEEauthorblockN{Simone Marrocco - 239951}
  \IEEEauthorblockA{simone.marrocco@studenti.unitn.it}
}

% make the title area
\maketitle

% As a general rule, do not put math, special symbols or citations in the abstract
% \begin{abstract}
%   TODO
% \end{abstract}


% For peerreview papers, this IEEEtran command inserts a page break and creates the second title. It will be ignored for other modes.
\IEEEpeerreviewmaketitle

\input{1.introduction.tex}
\section{Hidden comunication by targeted reflections}

We will start from the paper \textit{Reconfigurable Intelligent Surface: Reflection Design Against Passive Eavesdropping} \cite{9328149}, explaining how to hide communication between two actors from eavesdroppers using Reconfigurable Intelligent Surfaces, then expanding it to multiple receiving users at the same time.

\begin{figure}[H]
  \centering
  \includegraphics[width=\linewidth]{imgs/problem-description.png}
  \caption{Setup}
  \label{fig:correlation_sk}
\end{figure}

In \cite{9328149}, the authors studied how to use RISs to allow communications between two users without LOS, while making the signal undeciphrable for eavesdroppers. We call $L$ the transmitter's antennas, $K$ the receiver's antennas, $M$ the eavesdropper's antenna, and $N$ the RIS reflecting elements. We assume $L \ge K \ge 2$.

We define $H \in \C^{NxL}$ the channel response \footnote{A channel response for a MIMO communication is a matrix made of complex number, where the position $i,j$ indicates the signal received from antenna $j$ to antenna $i$} from the transmitter to the RIS, $G \in \C^{KxN}$ the channel response from the RIS to the receiver, $P = diag\{p\} \in \C^{NxN}$ a diagonal matrix in which the $n$th diagonal element represents the reflection coefficient of the $n$th unit at the RIS.

The objective is making the receiver's final signal $GPH$ a diagonal matrix, while making every possible eavesdropper's final signal a full matrix.

We will leave for later the technical details of why this would achieve secrecy for the legitimate users or how the actors communicate with each others, and will just focus on the mathematics behind the calculation. It is possible to read more in the paper \textit{Space shift keying modulation for MIMO channels} \cite{5165332}, which we will summarize in a later chapter.

Our contribution to the field will be to generalize these calculations to $J$ receving users and $M$ RISs used in parallel and in sequence.

Formally, the condition we want to satisfy is:

\begin{equation}
  || [GPH]_{:,1:K} - [[GPH]_{:,1:K}]_{diag} || ^2 = 0
\end{equation}

Where $[GPH]_{:,1:K} \in \C^{KxK}$ denotes the first K columns of the matrix $GPH \in \C^{KxL}$.

Given

\begin{equation}
  W = \sum_{i,j = 1, i \ne j}^{K} (g_{j,:} \odot h_i^T)^H (g_{j,:} \odot h_i^T)
\end{equation}
\begin{equation}
  rank(W) = K(K-1)
\end{equation}
\begin{equation}
  rank(W) - nullity(W) = N
\end{equation}
\begin{equation}
  nullity(W) = N - (K^2 - K)
\end{equation}

The formula (1) can be rewritten as

\begin{equation}Wp = 0\end{equation}

and the solutions of $p$ can be found in the null space of $W$. Using singular value decomposition (SVD), we can decompose

\begin{equation}
  W = R \Sigma V^H \footnote{An hermitian transpose of $V$ ($V^H$), means we fist transpose the matrix ($V \rightarrow V^T$), then take the conjugate of every element (so invert the sign of the immaginary part).}
\end{equation}

With SVD, we have $\Sigma = diag(\sigma) \in C^{NxN}$ a diagonal matrix. the first $rank(W) = K^2-K$ elements of $\sigma$ are non-zero, while the last $nullity(W) = N - (K^2-K)$ elements are zero \cite{svd}.

Given a more generic $A \in \C^{mxn} = R' \Sigma' V'^H$, we have the column vectors of $R'$ being an orthonormal span of $C^m$, and the row vectors of $V'$ being an orthonormal span of $C^n$.

Suppose $A$ is an Hermitian matrix (meaning $A = A^H$). This will be useful later, as $W$ is also an Hermitian matrix by construction. Let's call $k$ the null space dimension of $A$, and ,by the property above, the null space dimension of $A^H$ too.

The last $k$ columns of $R'$ are a span of the null space
\begin{equation}
  N(A^H) = [r_{m-k}, ..., r_m ] \in \C^{mxk}
\end{equation}
while the last $k$ rows of $V'^H$ are a span of the null space
\begin{equation}
  N(A) = \begin{bmatrix} v'^H_{n-k} \\ ... \\ v'^H_n \end{bmatrix} \in \C^{kxn}
\end{equation}
Being $A$ an Hermitian matrix, the two null spaces are both solutions to $Ax = 0$.

Consider now $W \in C^{NxN}$. The paper in question uses equation (7) to find the solutions, since $W$ is hermitian and square. Taken $U \in \C ^ {Nx(N-(K^2 - K))}$ the last $N-(K^2 - K)$ columns of the left singular matrix $R$. $U \in N(W)$ for the explanation above. We then have

\begin{equation}WU = 0\end{equation}
\begin{equation}p = Uq\end{equation}
\begin{equation}WUq = 0\end{equation}

being true, and $q \in \C^{N-(K^2 - K)}$ can be a random vector.

\section{Space Shift Keying Modulation}
But how can the actors communicate, if the result is a diagonal matrix with random value?

We will use a technique called \textit{Space Shift Keying} (SSK) Modulation \cite{5165332}, where \textit{antenna indices are used as the only means to relay information}. Given $K$ the number of antenna of the actors in the system, we can send $log_2(K)$ bits by mapping each combination of bits to a specific antenna.
\footnote{This may seem rather unoptimized, as we use only one antenna instead of combinations of them. To see a more general approach, the authors also wrote the paper \cite{4699782}, where they discuss a more general approach using multiple active antennas at the same time. The general approach will also work with our proposed solution.}

\begin{figure}[H]
  \centering
  \includegraphics[width=\linewidth]{imgs/ssk_conversion_table.png}
  \caption{SSK conversion table}
  \label{fig:ssk_conversion_table}
\end{figure}

\subsection{Direct Detection}
Given a channel gain matrix $B \in \C^{KxK}$ and the input vector $x \in \C^K$ with only one element equal to $1$, the signal received is
\begin{equation}
  y = Bx + \sigma^2
\end{equation}
To understand the antenna index which sent the message, we need to find the column $b_j$ which is most similar to $y$.

\begin{equation}
  j = arg\ max_j\ p_y (y | x_j, B) = arg\ min_j\ || y - b_j ||^2
  \label{eq:direct_detection}
\end{equation}

\subsection{Diagonalized Reflection Detection}
Following \cite{9328149}, for a reflected signal we have
\begin{equation}
  y = GPHx + \sigma^2
\end{equation}
Given that $GPH$ is a diagonal matrix and $x$ has only one element equal to $1$, the resulting vector $GPHx$ will still be a vector with only one element non zero. Adding noise, to find the antenna index we search for the biggest value in the vector.

\begin{equation}
  j = arg\ max_j\ y_j
  \label{eq:reflection_detection}
\end{equation}
\input{4.cascaded-channel-estimation.tex}
\input{5.ris-reflection-expansion.tex}
\input{6.simulation-results.tex}
\section*{Conclusion}

In this paper, we have expanded on the work presented in \cite{9328149} regarding Physical Layer Security using Reconfigurable Intelligent Surfaces (RISs). We generalized the framework to support multiple receiving users and multiple RIS configurations, both in parallel and in series. By mathematically proving the formulas, and physically simulating realistic scenarios, we demonstrated the validity and usefulness of the prooposed work.

With our contribution, the framework is now able to manage:
\begin{itemize}
  \item Multiple receivers in different positions
  \item Multiple RISs in parallel that increase signal quality and security
  \item Multiple RISs working in series to accomodate complex situation
  \item A wide combination of these properties in realistic network conditions
\end{itemize}

With our Bit Error Rate (BER) simulations, we proved and demostrated how the receivers are able to receive correctly the messages with a low error percenteage, while ensuring no other malicious actor can decypher the signal when not having direct Line of Sight (LOS) from the transmitter. Even when this link is present, our configurations ensure the RIS distrupt the interception of the signal with significant noise, even at high Signal to Noise Ratio (SNR).

We also showed the realistic application of our framework in a simulated scenario including realistic channel gain calculations, adding Rician fading and considering signal strenght using path loss. These added simulation will aid exporting our solution from a mathematical proof to an effectivle implementation usable for real life communication. We modeled different possibilities of path loss and RIS implementation to cover all possible variables, showing promising results even in the worst scenarios.

The implications of this work are particularly relevant for emerging technologies such as vehicular networks, Internet of Things, and other applications requiring secure wireless communications. Thanks to modern technologies, we are able to increase the security and privacy even at lower layers of communication, helping reducing the load on higher layers wich could impact negatively the usefulness of communications when latency and frequency of communication is crucial.

Future research directions could include:
\begin{itemize}
  \item Further optimization of RIS configurations for dynamic environments with mobile nodes
  \item Integration with existing security protocols at higher network layers
  \item Usage of more complex communication protocol, like GSSK \cite{4699782} instead of the proposed SSK \cite{5165332}
  \item Implementation and testing in real-world scenarios, particularly in vehicular networks
  \item Extension to even more complex network topologies with multiple transmitters and heterogeneous receiver capabilities
\end{itemize}

In conclusion, our extended framework for physical layer security using RISs provides a promising approach to secure modern wireless communication systems, especially in scenarios where traditional encryption methods may introduce unacceptable computational overhead or latency. The flexibility to support multiple users and complex reflection paths makes it adaptable to various practical deployment scenarios while maintaining strong security guarantees.


\bibliographystyle{IEEEtran}
\bibliography{references.bib}

\end{document}



