\documentclass[conference]{IEEEtran}
% \documentclass{report}

\usepackage{cite}
\usepackage{graphicx}
\usepackage{amsmath}
\usepackage{algorithmic}
\usepackage[caption=false,font=normalsize,labelfont=sf,textfont=sf]{subfig}
\usepackage{url}
\usepackage{hyperref}
\usepackage{float}
\usepackage{listings}
% Customizing lstlisting to mimic verbatim
\lstset{
  basicstyle=\ttfamily,
  breaklines=true, % Allows line breaks
  postbreak=\mbox{\space}, % Optional: mark line breaks
  escapeinside={(@}{@)}, % Define escape characters
}


\begin{document}
\title{TODO}


% author names and affiliations
% use a multiple column layout for up to three different
% affiliations
\author{
  \IEEEauthorblockN{Simone Marrocco - 239951}
  \IEEEauthorblockA{simone.marrocco@studenti.unitn.it}
}

% make the title area
\maketitle

% As a general rule, do not put math, special symbols or citations
% in the abstract
\begin{abstract}
  TODO
\end{abstract}


% For peerreview papers, this IEEEtran command inserts a page break and
% creates the second title. It will be ignored for other modes.
\IEEEpeerreviewmaketitle

\section{INTRODUCTION}
This is an example of a citation \cite{example_reference}.

\newpage
\bibliographystyle{IEEEtran}
\bibliography{references.bib}

\end{document}



