\documentclass[conference]{IEEEtran}
% \documentclass{report}

\usepackage{cite}
\usepackage{graphicx}
\usepackage{amsmath}
\usepackage{algorithmic}
\usepackage[caption=false,font=normalsize,labelfont=sf,textfont=sf]{subfig}
\usepackage{url}
\usepackage{hyperref}
\usepackage{float}
\usepackage{listings}
% Customizing lstlisting to mimic verbatim
\lstset{
  basicstyle=\ttfamily,
  breaklines=true, % Allows line breaks
  postbreak=\mbox{\space}, % Optional: mark line breaks
  escapeinside={(@}{@)}, % Define escape characters
}


\begin{document}
\title{TODO}


% author names and affiliations
% use a multiple column layout for up to three different
% affiliations
\author{
  \IEEEauthorblockN{Simone Marrocco - 239951}
  \IEEEauthorblockA{simone.marrocco@studenti.unitn.it}
}

% make the title area
\maketitle

% As a general rule, do not put math, special symbols or citations
% in the abstract
\begin{abstract}
  TODO
\end{abstract}


% For peerreview papers, this IEEEtran command inserts a page break and
% creates the second title. It will be ignored for other modes.
\IEEEpeerreviewmaketitle

\section{Paper summaries}

\subsection{Physical layer security in wireless networks: a tutorial}
Paper \cite{5751298} talks about how telecommunications need protections against active attacks, like jamming, and passive attacks, like eavesdropping or traffic analysis, and how general exchange of information could benefit from extra security coming from the physical layer.

To increase security, some solutions are proposed:
\begin{itemize}
  \item Radio Frequency Fingerprinting, where you extract feature from each signal to identify legitimate users and intruders;
  \item Error correction coding, to increase encryptions levels without losing relevant bits that could distrupt the decoding;
  \item Spread spectrum coding, sending the transmission in multiple different frequencies to increase success in case of jamming;
  \item Directional antennas, to avoid jammed paths (source 23, Guaranteeing Secrecy Using Artificial Noise);
  \item Artificial Noise Schemes, sending noise elsewhere to make the intruder intercept the signal impossible (source 24, Secret Communication in Presence of Colluding Eavesdroppers);
\end{itemize}

\subsection{The challenges facing physical layer security}
Paper \cite{7120011} talks about the problems we have when dealing with physical layer security and supporting authentication and confidentiality. We should think more about our assumptions when thinking how to protect our communications.

There are different hurdels in how we model the threats:
\begin{itemize}
  \item Passive adversaries, that only listen to the communication, but in reality they may interfere with the communication;
  \item The adversary may have different observation and better signal than the legitimate user;
  \item the adversary may have more resources to spend on computing power or better antennas;
  \item Simpler enviroment configurations could be more difficoult to protect;
\end{itemize}

\subsection{Counteracting Eavesdropper Attacks Through
  Reconfigurable Intelligent Surfaces: A New Threat
  Model and Secrecy Rate Optimization}
Paper \cite{10143983}

\newpage
\bibliographystyle{IEEEtran}
\bibliography{references.bib}

\end{document}



