\documentclass[conference]{IEEEtran}
% \documentclass{report}

\usepackage{cite}
\usepackage{graphicx}
\usepackage{amsmath}
\usepackage{algorithmic}
\usepackage[caption=false,font=normalsize,labelfont=sf,textfont=sf]{subfig}
\usepackage{url}
\usepackage{hyperref}
\usepackage{float}
\usepackage{listings}
% Customizing lstlisting to mimic verbatim
\lstset{
  basicstyle=\ttfamily,
  breaklines=true, % Allows line breaks
  postbreak=\mbox{\space}, % Optional: mark line breaks
  escapeinside={(@}{@)}, % Define escape characters
}


\begin{document}
\title{Phisical Layer Security for Vehicular Networks using Reconfigurable Intelligent Surfaces}


% author names and affiliations
% use a multiple column layout for up to three different
% affiliations
\author{
  \IEEEauthorblockN{Simone Marrocco - 239951}
  \IEEEauthorblockA{simone.marrocco@studenti.unitn.it}
}

% make the title area
\maketitle

% As a general rule, do not put math, special symbols or citations
% in the abstract
\begin{abstract}
  TODO
\end{abstract}


% For peerreview papers, this IEEEtran command inserts a page break and
% creates the second title. It will be ignored for other modes.
\IEEEpeerreviewmaketitle

\section{Introduction}
\subsection{Physical Layer Security}

In telecommunications, we have two type of threats we need to protect against. Active attacks, like jamming a frequency, distrupt and block the flow of information and we need to protect our communications, while passive attacks, like eavesdropping, are more subtle and we need to make our signals undeciphrable with encryption or noise \cite{5751298}. There are different methods we can use to mask our communications: we can fingerprint the legitimate users, add noise to our signal, spread it through multiple frequencies, use directional antennas or artificial noise schemes \cite{5751298}.

The adversaries may also have better resources, both in computer power and signal reception, and it is difficoult to model all possible threats we may face \cite{7120011}.

In particular to eavesdropping, there is a huge opprtunity for improvements. While distruptions have been studied for long, expecially in military communications, message encryption is usually delegated to the higher levels \cite{6739367}. However, the physical layer can assist by hiding or masking the signal, making it harder for the eavesdropper to capture the it. Given the advances in quantum computing and encryption breaking \cite{365700}, it is important to be protected at all layers.

Achieveing perfect communication secrecy in all cases is not really possibile for all cases, given that we need the secret key to be at least as big as the secret message \cite{6769090}, but there are some practical strategies we can implement.

In \cite{7543509} a statistical model is created to calculate the probability of archieving secrecy from eavesdroppers in unknown locations, while in \cite{4543070} and \cite{1605889} it is discussed how we can use the antenna spare power to induce artificial noise to assure the legitimate receiver has better signal.

\subsection{Reconfigurable Intelligent Surface}

With Reconfigurable Intelligent Surfaces (RIS), it is possible to control the propagation and reflection of signals, making it possible to transform the enviroment, in which the waves need to travel, from an uncontrollable phenomena to a programmable variable that is possible to (partially) control and optimize \cite{9086766}.

RISs can help in particular in two scenarios. In the first one, two nodes which are not in the line of sight (LOS) can communicate with the help of the RIS; in the second one, being in the LOS means an inability to take advantage of delayed reflections (especially for new tecnologies like 5G and 6G), which can be used to improve the signal quality and robustness, but we can create them with RISs \cite{9086766}.

The main advantages of RISs are the low cost, the low power consumption and the easy deployment, which makes them a good candidate for the future of wireless communications. They do not require a dedicated energy source, they do not suffer from noise amplification, they can work with any frequency and can be easly put in any surface like walls or ceilings \cite{8796365}.

\subsection{Using RISs for Physical Layer Security}
\subsection{RISs for vehicular networks}
\subsection{Physical Layer Security for Vehicular Networks}
\subsection{Challenges and Future Directions}

\newpage
\bibliographystyle{IEEEtran}
\bibliography{references.bib}

\end{document}



